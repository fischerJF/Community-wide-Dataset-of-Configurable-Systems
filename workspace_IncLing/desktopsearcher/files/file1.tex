\section{Introduction}

% SPLs are ...
Software product lines (SPLs) have been attracting an increasing
attention as an approach to improve software development and
maintenance~\cite{cmu-sei-website, software-product-line-engineering}.
An SPL is a family of related products, where each product is defined
by its specific combination of \emph{features} that represent
increments in functionality.  SPLs are commonly constructed using
\emph{feature variables}, which are boolean variables that guard the
inclusion of the functionality of the corresponding feature (in the
spirit of \CodeIn{ifdef} preprocessing
directives)~\cite{ahead-tool-suite, czarnecki-2000}---~setting and
unsetting feature variables allows constructing desired
products.\Comment{
% from Darko: it's better *not* to talk about this because we have no
% evaluation from Groupon or any other configurable system; so we'll
% be just SPL-specific
Thus, SPLs are a specific form of \emph{configurable systems}, which
have a fixed set of configuration \emph{parameters} (which are feature
variables for SPLs) that take values over finite domains (boolean for
SPLs), and initializing these parameters initializes the system state.
Such configurable systems are common in modern websites.  For example,
the \CodeIn{groupon.com} website has over \GrouponFeaturesNum{}
boolean configuration parameters and can, in principle, be deployed in
over $2^{\GrouponFeaturesNum{}}$ configurations.}

% SPLs are important ... % we could add text if we need to fill space!

% Feature models are ...
\emph{Feature models} formally capture dependencies among
features, i.e., they encode which combinations of features are \consistent{}
and which combinations of features are not.  If an SPL has $N$
(boolean) features, there are $2^N$ possible combinations, which we
also refer to as \emph{configurations}.  We call a feature model
\emph{\complete} if it assigns a \consistent{} or \inconsistent{} label to each of the
($2^N$) combinations.  In contrast, a feature model is
\emph{\incomplete} if it assigns an ``unknown'' label to some
combination; in the limit, a feature model is \emph{empty} if it
assigns an ``unknown'' label to all ($2^N$) combinations.\don{The last
  part of this paragraph is a bit strange.  A feature model that maps
  all combinations to unknowns seems useless. You need to specify or
  define ``complete'' and ``incomplete'' precisely} Formally, an
\incomplete{} feature model is a function from feature combinations to
the set $\{C,I,U\}$ (for \consistent{}, \inconsistent{}, and unknown combinations).

\Comment{
% from Darko: I don't know if this text was copied from somewhere
% (SPLat paper?!), so I'm not going to reuse any of this.  Please do
% not copy text from anywhere; we may end up with exact some text
% simply by naturally making our changes :), but we should *not* copy.
Testing product lines with many feature variables\Comment{ or configurable
systems with many configuration parameters} is particularly challenging
because each test needs to be run against potentially a combinatorial
number of products or configurations\footnote{For ease of exposition,
  we use the terms product line and configurable system
  interchangeably.}.  Fortunately, in practice, features often have
dependencies, termed \emph{feature models}, which constrain values of
feature variables and render only a subset of products as
\emph{\consistent{}}, thereby reducing the number of products a test needs to
be run against.  Unfortunately however, precise and complete feature
models may not be available in a formal language and thus a testing
technique may not be able to utilize them.  For example, the
\CodeIn{groupon.com} codebase has \GrouponFeaturesNum{} tests but no
formal feature model.
}

% There has been a lot of recent work on testing SPLs...
There has been a lot of recent research on testing
SPLs~\cite{kim-etal-fse2013, dagstuhl-seminar-spl-analysis-2013,
apel-icse2013, shi-etal-fase2012, charles-etal-icse2012,
kim-etal-issre2012, hwan-aosd-2011, Garvin-issre2011,
DBLP:conf/splc/CabralCR10}.
Testing an SPL systematically is a
challenging problem because it requires running a test suite against a
combinatorial number of products.  Feature models can, in principle,
play a key role in addressing this problem by (1)~reducing the space
of products to test and (2)~allowing accurate categorization of
failing tests as failures of the corresponding products or the tests
themselves, not as failures due to \inconsistent{} combinations of features.
However, \emph{all} cited prior work on testing SPLs assumes that the
system under test comes with a \complete{}, formally specified feature
model; the work focused on speeding up testing in such cases.

In practice, unfortunately, the case of \incomplete{} feature models
is common as most systems lack a \complete, formally specified feature
model.  Without a \complete{} feature model, existing techniques can
generate false alarms, where a test fails for some feature combination
because the combination is \inconsistent{}, not because the test code or the
code under test is buggy.  Indeed, the absence of a \complete{}
feature model presents a basic problem for testing an SPL: a test
failure does not necessarily indicate a fault in the product under
test because the test itself may be erroneous or the product it tests
may simply be \inconsistent{}, i.e., disallowed by the feature model that is
not fully available.

To illustrate\Comment{ on a specific practical example}, partial motivation for
this work stems from the project~\cite{kim-etal-fse2013} we performed
on the Groupon codebase\footnote{Unfortunately, we have no access to
the Groupon codebase any more since Darko Marinov finished his visit
to Groupon a few months ago.}.  Groupon had over \GrouponTestsNum{}
tests for its main web application.  Most of those tests were written
to work for only one configuration (most often the default reference
system configuration that assigns certain concrete values---true or
false---to each of the over \GrouponFeaturesNum{} boolean features).
We realized that there could be a lot of value in running those tests
on other configurations, even if the tests were not written for those
configurations.  Indeed, using SPLat~\cite{kim-etal-fse2013} we ran
these tests\Comment{ on a total of over \Fix{40K} configurations and got over
\Fix{4K} test failures} and executed tens of thousands configurations
with several thousands failures.  However, Groupon had no formal feature model,
so we could not easily tell if these failures were due to \inconsistent{}
combinations of features or due to \consistent{} combinations of features that
revealed some real problems with tests or code.  \Comment{There was no formal
feature model due to the difficulty of maintaining it when software
quickly evolves as is common for web applications.}\don{I'm not sure
about the last statement: please remove.}\divya{Done.}

% There has been also work on inferring feature models...
% We should sprinkle here that this is a hard and challening problem,
% and no one solves it fully automatically, so it's okay that we do
% something manual later on.
There has been also recent research on inferring/extracting feature models
from various software artifacts~\cite{she-etal-icse2011,
alves-splc-2008, davril-etal-fse2013, haslinger-etal-fase2013,
andersen-etal-splc2012, lopes-herrejon-etal-ssbse2012, acher-vamos2012,
weston-splc2009, czarnecki-wasowski-splc2007}.  For example,
She~\etal{}~\cite{she-etal-icse2011} use a static analysis to extract
feature dependencies from code, make files, comments, and
documentation.  The technique is not fully automatic; the user needs
to select, for each feature, its feature parent from a list of options
provided by the infrastructure.  Alves~\etal{}~\cite{alves-splc-2008}
and Davril~\etal{}~\cite{davril-etal-fse2013} use information
retrieval and data mining.  Alves~\etal{}~\cite{alves-splc-2008} and
Haslinger~\etal{}~\cite{haslinger-etal-fase2013} assume that the user
provides a list of \emph{all} \consistent{} configurations.
Lopez-Herrejon~\etal{}~\cite{lopes-herrejon-etal-ssbse2012} use
evolutionary search.
% what do these ``evolutionarist'' take as input?!
However, \emph{no} cited prior work on inferring feature models
exploits tests and their executions.\don{$\Leftarrow$Strange organization--this is
  related work or should be in related work section.}

This paper bridges the gap between (1)~testing SPLs requiring
\complete{} feature models and (2)~inferring feature models not using
information from tests.\divya{shouldn't the second point be \emph{inferring feature models Using
information from tests}}  We introduce
\emph{\tname}\Comment{\footnote{\underline{SPL} testing from
\underline{i}ncomplete \underline{f}eature models}}, a novel technique for
more effective testing of SPLs that does not require a priori availability of
\complete{} feature models.  To the best of our knowledge, this is the
first such technique.  Our key insight is that by running each test
from a given test suite against many products we can utilize the
information from failing and passing runs to help developers
prioritize their inspection of failures to accurately distinguish
failures in code (including tests) from failures due to \inconsistent{}
combinations of features.

\don{My writing style comment: too much for an introduction.}In brief,
\tname{} proceeds as follows.  Given a suite of product line tests,
\tname{} first uses dynamic reachability analysis to run each test
against several products, following our recently proposed SPLat
technique~\cite{kim-etal-fse2013}.  Next, \tname{} ranks the failing tests
 according to their likelihood of indicating a product
failure as opposed to\Comment{vs.} an \inconsistent{} configuration.  Finally, \tname{} ranks the
configurations according to their likelihood of being \consistent{}
configurations of features.  The user can then inspect the ranked
configurations and tests (1)~to determine whether the test failures
are due to \inconsistent{} configurations or bugs in the SPL code or tests (in
which case the code and tests can be updated) and (2)~to update the
feature model to make it more \complete.  \tname{} can then reuse the
new information to rerank the tests and configurations for further
inspection.

In sum, this paper makes the following contributions:

%\newenvironment{Contributions}{\begin{itemize}}{\end{itemize}}
%\newcommand{\Contribution}[1]{\item \textbf{#1:}}

\newenvironment{Contributions}{}{}
\newcommand{\Contribution}[1]{\vspace{0.5em}\noindent$\bullet$\hspace{0.5em}\textbf{#1:}}

\begin{Contributions}
\setlength{\itemsep}{1pt}
\setlength{\parskip}{0pt}
\setlength{\parsep}{0pt}

\Contribution{Technique} We present \tname, the first approach that
synergistically exploits tests and \incomplete{} feature models (in
the limit, starting from \Empty{} feature models) to help users both
(1)~\Comment{prioritize inspection of failing test runs}\divya{quickly identify test failures caused by real faults in product code distinguishing them from those caused by inconsistent configurations} and (2)~build a more
\complete{} feature model.\don{Notion of an incomplete feature model
  is novel in itself. Explain how it is represented and how it works.
  This is not clear me.}

\Contribution{Implementation} We have implemented our \tname{}
technique for Java SPLs.  Our implementation builds on our SPLat
code~\cite{kim-etal-fse2013, SPLatWebSite} for exploring SPL tests on
various reachable configurations.

\Contribution{Evaluation} We have evaluated \tname{} on \numsubjects{} SPLs.
The experimental results demonstrate the utility of \tname{} in
automating testing of product lines with \incomplete{} feature models.

\end{Contributions}

\Comment{
%% \mar{Sabrina, could you please inform (from Darko: what does ``inform'' mean in Brazilian Portguese?!) pages from C. Pohl's book on
%%   different strategies for writing tests for SPLs. Are we assuming
%%   some strategy in particular?}  \sab{The pages: 271 to 281. If I
%%   understood well, our tests (made for one product) fits on the second
%%   strategy: Pure application strategy}
In contrast to single-system, testing activities in product line have
to consider product line variability. So, Pohl \etal{}~\cite{software-product-line-engineering}
define four test strategies to deal with variability:
\begin{enumerate}

\item \textbf{Brute Fore Strategy}: takes into account all products for
create and perform tests. In this case, tests cannot have "assume" and
"assert", because tests have to run for all \consistent{} configurations;

\item \textbf{Pure Application Strategy}: just consider one product
(application) and only product-specific tests are created. In this case,
the test must have "assume" because it is specific for a configuration,
and It cannot have "assert", because it is already specific for a assumed
configuration, it is not expected It works for other configurations;

\item \textbf{Sample Application Strategy}: uses one or a few sample
products to test, and the tests are specific for these products. In this
case, the test must have "assume" because it is specific for a set of
configurations, and It also must have "assert", because it is expected
that it works for the assumed set of configurations;

\item \textbf{Commonality and Reuse Strategy}: aims at reuse tests for
common parts and prepare tests for variable parts. In the SPL code we have
part of the code that is common for all products (commonality) and other
parts that vary according to the feature it belongs to. So, this strategy
proposes to reuse tests for the common parts and create specific tests for
variable parts. In this case, tests are not focusing in configurations, but
in commonalities and variabilities. So, for commonalities, tests cannot have
"assume" and "assert", because behavior is general for all products. And, for
variabilities, tests must have "assume" and "assert", because behaviors are
related to features.

\end{enumerate}
%% The authors point that those strategies should be used in combination, and they
%% consider the combination of (3) and (4) to be the best.
}

%% \mar{What do we want to do with feature models (FMs)?  Are we going to
%%   motivate the need of FMs as in other papers?  For example, to help
%%   understanding, find contradiction between code and spec, and support
%%   product generation? Or the users do not need to be aware of the FMs
%%   we extract?} We have recently proposed
%% \splat{}~\cite{kim-etal-fse2013}, a technique to execute a test case
%% against\Comment{ all} dynamically reachable system configurations.
%% Although experimental results indicate that \splat{} is promising in
%% its purpose of reducing execution costs, \splat{} assumes that the
%% test case is adequate for all configurations it can reach.
%% Unfortunately, that is a strong assumption in practice.  A test run
%% can fail for a configuration $c$ for the following reasons:

%% %%\begin{enumerate}[(i)]\itemsep1pt
%% \begin{enumerate}

%% \item Configuration $c$ is \inconsistent{}; results are unpredictable.  It is
%%   not possible to reject configurations of this kind under the
%%   scenario where feature models are not documented.

%% \item The configuration $c$ is \consistent{}.  Two alternatives:

%%   \begin{itemize}\itemsep-1pt
%%   \item The test fails because it is not general enough to run on $c$.
%%     For example, it is necessary to assert that some specific feature
%%     is required and/or to specialize or generalize assertions and
%%     parts of the test sequence.
%%   \item The test fails because there is a bug in the
%%     system.
%%   \end{itemize}

%% \end{enumerate}

%% This paper presents a technique that enables the tester to rule out
%% option ``i''.\mar{are we going to say that our technique adds
%%   precondition checks to existing tests?}  Consequently, in the
%% presence of a failure report, she needs to decide only between two
%% alternatives: the test indeed reveals a bug or it needs
%% repair.\Comment{ \sab{what kind of repair?}\mar{it is mentioned in the
%%     bulleted list.  please check.}}  Our technique distinguishes
%% option ``i'' by automatically discovering an approximated model that
%% characterizes \consistent{} system configurations.  It learns such model from
%% a sample of test runs and only reports to the user failures that are
%% due to likely \consistent{} configurations.  \mar{novelty in comparison to
%%   previous work...}

% LocalWords:  Groupon codebase pre Pohl's SPLs SPL combinations boolean Alves
% LocalWords:  combinatorial Davril Haslinger Herrejon ncomplete eature priori
% LocalWords:  reachability SPLat rerank synergistically groupon Pohl ifdef
% LocalWords:  variabilities preprocessing unsetting Darko Marinov
